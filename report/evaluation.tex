\section{Evaluation}

The evaluation consists of two parts: one with computer-generated testing set, one with hand-drawn sketches.

\subsection{Computer-generated Testing Set}

\subsubsection{Result on Developing Set}

The confusion matrix for the evaluation on the developing set is in Table 1. The developing set contains 1000 figures with 200 figures from each category.

\begin{table}[ht!]
\centering
\begin{tabular}{|l|l|l|l|l|l|}
\hline
\backslashbox{Label}{Recognized} & Circle & Ellipse & Triangle & Square & Rectangle \\ \hline
Circle & \textbf{0.55} & 0.03 & 0.42 & 0.00 & 0.00 \\ \hline
Ellipse & 0.12 & \textbf{0.82} & 0.06 & 0.00 & 0.00 \\ \hline
Triangle & 0.00 & 0.00 & \textbf{0.99} & 0.00 & 0.01 \\ \hline
Square & 0.00 & 0.00 & 0.03 & \textbf{0.79} & 0.19 \\ \hline
Rectangle & 0.00 & 0.00 & 0.04 & 0.39 & \textbf{0.56} \\ \hline
\end{tabular}
\caption{Confusion Matrix on Developing Set}
\end{table}

\subsubsection{Error Analysis and Tuning}

First we see that nearly a half of circles are incorrectly recognized as triangles. Upon examination we find that these circles actually do not fall into any fuzzy sets because of their larger-than-expected $Thinness$ ratio. Sorting on the list with each shape's score = 0, a unexpected side effect happen that the letter ``T'' is alphabetically the largest. This problem can be fixed by making sure that \textbf{all fuzzy sets are interleaved and no blank area is left}, as what we do in this case of circles vs. squares.

Second we see that only 82\% ellipses are correctly recognized. The issue here is similar to the first one, but slightly different with the solution. The $Extent$ ratio for triangles and rectangles are quite centralized around 0.5 and 1.0. Therefore we can \textbf{safely expand the confidence range} of ellipses.

Finally we see that squares and rectangles are sometimes mixed up. This makes sense because \textbf{some of rectangle images in our data do look like squares}. If we are to deal with this type of error, we may consider using new features, such as identifying and comparing all the sides in the shape.

Besides, there are about 5\% figures showing a $Extent$ ratio of 1 due to image processing errors. Then when these figures happend to be non-rectangles, they will be recognized as triangles as explained above. The reason for the errors are currently unknown.

\subsubsection{Result on Testing Set}

After applying the necessary tweaks to the ``parameters'' (boundaries of fuzzy sets), we evaluate our system on the testing set, which contains 1000 figures with 200 figures from each category. The confusion matrix is in Table 2.

\begin{table}[ht!]
\centering
\begin{tabular}{|l|l|l|l|l|l|}
\hline
\backslashbox{Label}{Recognized} & Circle & Ellipse & Triangle & Square & Rectangle \\ \hline
Circle & \textbf{0.96} & 0.01 & 0.03 & 0.00 & 0.00 \\ \hline
Ellipse & 0.08 & \textbf{0.90} & 0.02 & 0.00 & 0.00 \\ \hline
Triangle & 0.00 & 0.00 & \textbf{0.99} & 0.00 & 0.01 \\ \hline
Square & 0.00 & 0.00 & 0.00 & \textbf{0.90} & 0.10 \\ \hline
Rectangle & 0.00 & 0.01 & 0.03 & 0.39 & \textbf{0.58} \\ \hline
\end{tabular}
\caption{Confusion Matrix on Testing Set}
\end{table}

Apart from the squares and rectangles mixing up, we notice that circles and ellipses also mix up due to similar reasons: some of ellipse images in our data do look like circles. But as this is largely due to the \textbf{inherent ambiguity in the images we use}, it would not be a serious issue when testing against hand-drawn sketches. Because \textbf{sketches are usually exaggerated in some ways to avoid multiple interpretations}, unless the person who draws them intentionally make them very ambiguous or such a distinction is not necessary.

\subsection{Hand-drawn Sketches}

As the final evaluation of our system, we feed about 55 sketches drawn by two people to the recognizer. The sketches include 1 $\sim$ 3 figures each, and contain in total 20 figures for every category. The confusion matrix is in Table 3.

\begin{table}[ht!]
\centering
\begin{tabular}{|l|l|l|l|l|l|}
\hline
\backslashbox{Label}{Recognized} & Circle & Ellipse & Triangle & Square & Rectangle \\ \hline
Circle & \textbf{20} & 0 & 0 & 0 & 0 \\ \hline
Ellipse & 1 & \textbf{19} & 0 & 0 & 0 \\ \hline
Triangle & 0 & 0 & \textbf{20} & 0 & 0 \\ \hline
Square & 0 & 0 & 0 & \textbf{18} & 2 \\ \hline
Rectangle & 0 & 0 & 0 & 5 & \textbf{15} \\ \hline
\end{tabular}
\caption{Confusion Matrix on Sketches}
\end{table}

We can see that performance is pretty good. There's only one problem left: sometimes squares are recognized as rectangles. The reason seems to be that value of $Thinness$ ratio is somewhat very sensitive to the input image.
