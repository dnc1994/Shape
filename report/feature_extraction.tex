\section{Feature Extraction}

Feature extraction is mainly done with OpenCV. 

\subsection{Contours}

Contours are probably the most starightforward feature one can think of. With the help of OpenCV, once we've found the \textbf{contour} of the target figure, we can calcuate the following:

\begin{itemize}
\item Perimeter: apply \textit{cv2.arcLength} on the contour gives us an approximation of the perimeter.
\item Area: apply \textit{cv2.contourArea} on the contour gives us an approximation of the area.
\end{itemize}

\subsection{SIFT}

SIFT allows us to easily find key points (corners) in the image. At first glance it seems to be a great feature, since it can distinguish circles / ellipses from triangles and rectangles / squares. However, the extraction of this feature is \textbf{highly unstable}. A little noise (e.g. a zigzag line) in the image could increase the number of key points by more than 20. Although it's possible to set some thresholds to tune out the undesired key points, still we couldn't put too much confidence on this feature.

\subsection{Feature Expansion}

In computer vision, when it comes to geometric figures, there are several frequently used features that we can easily obtain using the existing features. Some of them turn out to be useful for our task:

\begin{itemize}
\item $Thinness = \frac{P^2}{A}$: Thinness measures the degree to which the figure is ``economical'' in using its perimeter to enclose an area. 
\item $Extent = \frac{A}{A_{br}}$: Extent measures the how much space the figure takes up in its bounding rectangle (the rectangle of the smallest area that bounds the target figure).
\end{itemize}

\subsection{Autonomous Feature Discovery}

We will discuss the possibility of autonomous rule derivation in Section 7.
