\section{Feature Extraction}

Feature extraction is mainly done with OpenCV. 

\subsection{Contours}

Contours are probably the most starightforward feature that one can be think of. With the help of OpenCV, once we've found the contour of the target figure, we can calcuate the following:

\begin{itemize}
\item Perimeter: apply cv2.arcLength on the contour gives us an approximation of the perimeter.
\item Area: apply cv2.contourArea on the contour gives us an approximation of the area.
\end{itemize}

\subsection{SIFT}

SIFT allows us to easily find keypoints(corners) in the image. At first glance it seems to be a great features, since it can distinguish circle/ellipse from triangles and rectangles/squares. However, the extraction of this feature it's highly unstable. A little noise in the image could increase the number of keypoints by, like, 20. Although it's possible to set some thresholds to tune out the undesired keypoints, still we couldn't put too much confidence on this feature.

\subsection{Feature Expansion}

In computer vision, when it comes to geometric figures, there are several frequently used features that we can easily obtain using the existing features. Some of them turn out to be useful for our task:

\begin{itemize}
\item $Thinness = \frac{Perimeter^2}{Area}$: Thinness measures the degree to which the figure is 'economical' in using its perimeter to enclose an area. 
\item $Extent = \frac{Area}{Area of bounding rectangle}$: Extent measures the how much space the figure takes up in its bounding rectangle (the rectangle of the smallest area that bounds the target figure).
\end{itemize}

Technically, we can generate more features based on current ones, as long as they are useful for in our task. Currently features are selected based on geometric knowledge (maybe ``common sense'' is a better word) and observation of the training data. It is entirely possible to apply some machine learning techniques to autonomously generate and select new features. We will discuss this in Section 7.
