\section{Rule Derivation}

\subsection{Manual Rule Derivation}

After observing a couple of patterns from data in the development set and reasoning about their validity, we come up with several heuristics.

First we notice that $Extent$ is a very good feature to start with: triangles have a $Extent$ ratio close to 0.5, rectangles and squares have a $Extent$ ratio close to 1, while circles and ellipses have a $Extent$ ratio of around 0.75 $\sim$ 0.85.

Then we use $Thinness$ ratio to distinguish circles and ellipses. Among common 2D figures, circle has the smallest thinness ($=4\pi$). Similarily, when rectangles have $Thinness$ ratio close to 16, they become squares.

Finally, we obtain 5 simple rules:

\begin{itemize}

\item IF Thinness IS LIKE Circle AND Extent IS LIKE ellipse THEN Shape IS Circle
\item IF Thinness IS NOT LIKE Circle AND Extent IS LIKE ellipse THEN Shape IS Ellipse
\item IF Extent IS LIKE Triangle THEN Shape IS Triangle
\item IF Thinness IS LIKE Square AND Extent IS LIKE rectangle THEN Shape IS Square
\item IF Thinness IS NOT LIKE Square AND Extent IS LIKE rectangle THEN Shape IS Rectangle

\end{itemize}

\subsection{Autonomous Rule Derivation}

We will discuss the possibility of autonomous rule derivation in Section 7.